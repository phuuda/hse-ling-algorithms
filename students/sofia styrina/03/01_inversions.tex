%% LyX 2.1.1 created this file.  For more info, see http://www.lyx.org/.
%% Do not edit unless you really know what you are doing.
\documentclass[english]{article}
\usepackage[T1]{fontenc}
\usepackage[latin9]{inputenc}
\usepackage{amstext}
\usepackage{babel}
\begin{document}

\section*{Problem 1}

\[
\pi(i,j):\quad i<j,\quad\pi_{i}>\pi_{j}
\]


Inversions in $\pi$

n - number of elements,

\[
\text{number of inversions}\le\frac{n(n-1)}{2}
\]


Proof by induction:

$n=2$, $\pi=(\pi_{1},\pi_{2})$ 

W.L.O.G. assume that $\pi_{1}<\pi_{2}$. Then the number of inversions
in $(\pi_{1},\pi_{2})$ equals 0, and the number of inversions in
$(\pi_{2},\pi_{1})$ equals $1=\frac{2\times(2-1)}{2}$.

If $\pi_{1}=\pi_{2}$, then the number of inversions in $(\pi_{1},\pi_{2})$
and $(\pi_{2},\pi_{1})$ equals zero.

\bigskip{}


Therefore, for n = 2, number of inversions $\le\frac{n(n-1)}{2}$

\bigskip{}


$n=3$, $\pi=(\pi_{1},\pi_{2},\pi_{3})$, number of inversions = 0.

$(\pi_{1},\pi_{3},\pi_{2}),$ number of inversions = 1.

$(\pi_{2},\pi_{1},\pi_{3})$, number of inversions = 1.

$(\pi_{2},\pi_{3},\pi_{1})$, number of inversions = 2.

$(\pi_{3},\pi_{1},\pi_{2})$, number of inversions = 2.

$(\pi_{3,}\pi_{2},\pi_{1})$, number of inversions = $3=\frac{3(3-1)}{2}$.

\bigskip{}


Consider $\pi=(\pi_{1},\pi_{2},\ldots,\pi_{n})$. Suppose that the
statement holds for $n-1$. I want to prove that it holds for $n$
as well. There are $n-1$ elements among $(\pi_{2},\ldots,\pi_{n})$.
The number of inversions in $(\pi_{2},\ldots,\pi_{n})$ is therefore
$\le\frac{(n-1)((n-1)-1)}{2}=\frac{(n-1)(n-2)}{2}$

\bigskip{}


Then, the number of in versions in $(\pi_{1},\pi_{2},\ldots,\pi_{n})\le\frac{(n-1)(n-2)}{2}+(n-1)$,
with $(n-1)$ being the maximum additional number of inversions that
can occur, if $\pi_{1}$ is greater than each of the $\pi_{2},\ldots,\pi_{n}$.

\bigskip{}


$\frac{(n-1)(n-2)}{2}+(n-1)=\frac{(n-1)(n-2)+2(n-1)}{2}=\frac{(n-1)(n-2+2)}{2}=\frac{n(n-1)}{2}$

\bigskip{}


Q.E.D.

\bigskip{}


Therefore, the number of inversions $=\frac{n(n-1)}{2}$, when all
the elements are placed in strictly descending order.
\end{document}
